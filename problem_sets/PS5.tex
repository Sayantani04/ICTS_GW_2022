% This is a template for doing homework assignments in LaTeX

\documentclass{article} % This command is used to set the type of document you are working on such as an article, book, or presenation

\usepackage[margin=1in,footskip=0.25in]{geometry} % This package allows the editing of the page layout
\usepackage{amsmath}  % This package allows the use of a large range of mathematical formula, commands, and symbols
\usepackage{graphicx}  % This package allows the importing of images
\usepackage{hyperref}
\usepackage{physics}

\newcommand{\question}[2][]{\begin{flushleft}
        \textbf{Problem #1}: %\textbf{Gaussian - Gaussian Model}  
\end{flushleft}
}

\newcommand{\sol}{\textbf{Solution}:} %Use if you want a boldface solution line
\newcommand{\maketitletwo}[2][]{\begin{center}
        \Large{\textbf{Problem Set - 4}
            
            Compact Binary Evolution, Rates and Population Modelling} % Name of course here
        \vspace{10pt}
        
        \normalsize{Aditya Vijaykumar  % Your name here
        } \\
        \vspace{5pt}  June 8, 2022     % Change to due date if preferred
        \vspace{15pt}
        
\end{center}}
\begin{document}
    \maketitletwo[5]  % Optional argument is assignment number
    %Keep a blank space between maketitletwo and \question[1]
    
    \question[1]{} \textbf{Review of LVK Results:} 
    Download posterior samples of the Power Law + Peak Model from \href{https://drive.google.com/drive/folders/1lBTo12EoWYjvR0Bf76PSy6tP2ms6nt41?usp=sharing}{this link}.
    \begin{enumerate}
    \item Plot the corner plot of the posterior samples. 
    
    \item Use the GWTC-3 paper to figure out the priors on these parameters and plot them too. Are the posteriors sufficiently different from the priors?
    
    \item What are the relevant degeneracies that you can see? Explain all of them.
    
    
    \item What statements about the underlying population of binaries can you make from these results?
    \item Select one sample at random. From this, create a distribution of the source frame mass m1. Draw a population of binaries from this distribution.
    \item \textbf{[Bonus]} Now repeat the above (still with one sample) to draw a population in mass ratio and redshift as well
    \item \textbf{[Bonus]} Use all the posterior samples to draw a population in all parameters.
\end{enumerate}
\end{document}